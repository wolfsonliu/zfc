\documentclass[11pt, a4paper]{article}

\usepackage{natbib} %a package for better cite \citep for parenthetical (author, 0000) \citet for textual author (0000)
\bibliographystyle{abbrvnat}
\bibpunct{}{}{,}{s}{}{\textsuperscript{,}}
\renewcommand\bibnumfmt[1]{#1.}


\title{Algorithm}

\begin{document}
\maketitle

The ZFC analysis algorithm adopts the z-score of log2 fold change as
the judgement of the sgRNA and gene changes between reference group
(without treatment) and experiment group (with treatment). ZFC
supports screening with iBAR \citep{zhu_guide_2019} employed, as
well as conventional screening with replicates. The sgRNA with
replicates and sgRNA$^{iBAR}$ is treated with similar procedure.

\section{Normalization of raw counts}

We use total counts for the normalization of raw counts, to rectify
the batch sequencing deptch. Because some sgRNAs in the reference have
very low raw counts, which can affect the fold change calculation of
the following analysis. We define sgRNAs counts less than 0.05
quantile both in reference group and experiment group as the small
count sgRNAs. The mean counts of all the small count sgNRAs were added
to the normalized counts. The normalized counts for sgRNAs in
reference group are calculated as following expression:

$$Cn_{i} = \frac{Cr_{i}}{S_{ref}} \times 0.5 \times (S_{ref} + S_{exp})$$

$$C_{i} = Cn_{i} + Cm_{small}$$

where, $Cn_{i}$ is the normalized count of ith sgRNA$^{iBAR}$ in
reference group, $Cr_{i}$ is the raw count of ith sgRNA$^{iBAR}$,
$Cm_{small}$ is the mean counts of all the small count sgRNAs,
$S_{ref}$ is the sum of raw counts of all the sgRNA$^{iBAR}$ in
reference group, $S_{exp}$ is the sum of raw counts of all the
sgRNA$^{iBAR}$ in experiment group, $C_{i}$ is the final normalized
counts for ith sgRNA$^{iBAR}$ after small count adjustment. The
normalized counts for sgRNAs in experiment group are calculated
similarly.

\section{Calculate fold change}

The raw fold change of each sgRNA$^{iBAR}$ is calculated from the
normalized counts of reference and experiment groups.

$$fc_{i} = \frac{Cref_{i}}{Cexp_{i}}$$
$$lfc_{i} = \log_{2}fc_{i}$$

where, $fc_{i}$ is the fold change (FC) of ith sgRNA$^{iBAR}$ and
$lfc_{i}$ is the log2 fold change (LFC) of ith sgRNA$^{iBAR}$,
$Cref_{i}$ and $Cexp_{i}$ are the normalized counts of reference
and experiment groups respectively.

\section{Calculate fold change standard deviation}

In order to calculate z-score of LFC (ZLFC), the standard deviation of
LFC should be calculated. The LFC of sgRNA$^{iBAR}$ is related to the
normalized counts of reference group. So the standard deviations of
LFC are different for sgRNA$^{iBAR}$s with different normalized counts
of reference group. All the sgRNA$^{iBAR}$s are divided into several
sets according to the normalized counts of reference group. And the
standard deviations of log fold change are calculated among the
divided sets. The LFC standard diviation and the normalized counts of
the reference is linearly related. So, a linear model is calculated
for the LFC sd and reference counts. And the linear model is used to
calculate the LFC standard diviation for all the sgRNA$^{iBAR}$s.

\section{Raw z score of log fold change}

The raw z score of log fold change is calculated.

$$raw ZLFC = \frac{LFC}/{LFC std}$$

\section{Remove sgRNA$^{iBAR}$ ZLFC with large leverage}

The leverage of the Raw ZLFC of sgRNA$^{iBAR}$ to the sgRNA mean ZLFC is
calcualted to distinguish the possible free-rider
sgRNA$^{iBAR}$s. sgRNA$^{iBAR}$s ZLFC with leverage larger than the
threshold are removed.

$$Leverage_{i} = \frac{1}{n} + \frac{(x_{k} - \={x})^{2}}{\sum_{k}^{n}(x_{k} - \={x})^{2}}$$

where `$Leverage_{i}$` is the `$i$th` sgRNA$^{iBAR}$ for one sgRNA, `$n$`
is the number of sgRNA$^{iBAR}$s for the sgRNA, `$\={x}$` is the mean of
the sgRNA$^{iBAR}$s' ZLFC.


% In order to modify the high leverage sgRNA$^{iBAR}$, we calculated the
% modification ratio by the leverage. The leverage ratio for dropout
% direction sgRNAs is calcualted as follow:

% $$Dropout Leverage Ratio_{i} = n \times \frac{2 - Leverage_{i}}{2 * n - 2}$$

% The raw ZLFC of one sgRNA$^{iBAR}$ is multiplied by the ratio to generate
% the modified ZLFC.

% And for the enrichment sgRNAs, free-rider is a serious problem, so the
% the leverage ratio is modified with a factor, $\sqrt{n}$, for higher punishment.

% $$Enrichment Leverage Ratio_{i} = \sqrt{n} \times \frac{2 - Leverage_{i}}{2 * n - 2}$$


\section{Calculate zscore of fold change p value in normal distribution}

Calculate sgRNA$^{iBAR}$ ZLFC p value from normal distribution.


\section{Calculate gene mean zscore of fold change}

The gene level ZLFCs are calculated as the mean of all the ZLFCs of
the relevant sgRNA$^{iBAR}$s. Considering the more sgRNAs for a gene
implying higher confidence on the result, a factor $\sqrt{n}$ is
multiplied to the mean ZLFCs.

$$ZLFC_{gene} = \frac{\sum{ZLFC_{sgRNA-iBAR}}}{n} \times \sqrt{n}$$

Empirical P value is also calculated for the gene ZLFCs. The p value
is adjusted considering control of False Discovery Rate
\citep{benjamini_controlling_1995}.


\section{Robust rank aggregation analysis}


Robust rank aggregation \citep{kolde_robust_2012} is utilized to
calculate the rank significance of the gene with the sgRNA$^{iBAR}$s
in the whole library. Aside from robust rank aggregation, mean rank
aggregation is also calculated. The robust rank score is adjusted
considering control of Fault Discovery Rate
\citep{benjamini_controlling_1995}.


\bibliography{al}

\end{document}


%%% Local Variables:
%%% mode: latex
%%% TeX-master: t
%%% End:
